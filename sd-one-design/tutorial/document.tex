\documentclass[]{book}
\title{Tutorial}
\author{Abdul Waajid}
\date{10th of october 2024}
%hypperref setup.

\usepackage{hyperref}
\hypersetup{
    colorlinks=true,
    linkcolor=blue,
    filecolor=magenta,      
    urlcolor=cyan,
    pdfpagemode=FullScreen,
    }
%hypperref setup.

\usepackage{flowchart}
\urlstyle{same}
\usepackage{parskip}
\usepackage{amsmath, amssymb, amsfonts}
\usepackage{graphicx}
\usepackage[top=1in,bottom=1.0in,left=0.5in,right=0.5in]{geometry}
\usepackage{enumerate}

\usepackage{float}

\setlength{\parskip}{0.3 em}
\setlength{\parskip}{1em}  % Add 1em space between paragraphs
\setlength{\parindent}{0pt} % Disable paragraph indentation


\begin{document}
\maketitle
\section{Tutorial one}
\begin{enumerate}
    \item Write an algorithm and flow chart for calculating the perimeter and surface area of a square, if the length of the sides of the square is given by the user.

        

    \item Write an algorithm and flow chart for a program, that will ask the user for his name, then print out "HI" and user's name. Then the program 
        will ask the user for his year of birth, then calculate his age and finally print out the age.

    \item Create an algorithm and flow chart that will recive an integer input, then add 5 to it, then double the resulting value, then procees to substract 7 
        from the resulting value, and finally display the computed value.

    \item  Company ABC requires a weekly payroll report for its salesmen. To achieve this,
        an algorithm and flow chart should be developed that takes the salesperson's
        name, number, and weekly sales as inputs.  The program will output the salesperson's name, number, and total pay. 
        Each salesman receives a base weekly pay of \$300, along with a commission of 10\% on their total 
        weekly sales. This algorithm and flow chart will allow the company to efficiently calculate and report the weekly earnings of its sales staff.

    \item Construct an algorithm and flow chart to read in three values from a customer’s bank account: the account
    balance at the beginning of the month, a total of all withdrawals from the account for the month,
    and a total of all deposits into the account during the month. A federal tax charge of 1\% is
    applied to all transactions made suing the month. The program is to calculate the account
    balance at the end of the month by.


    \begin{enumerate}

        \item a. Subtracting the total withdrawals from the account balance at the beginning of the month, 
        \item b. Adding the total deposits to this new balance,
        \item c. Calculating the federal tax (1\% of total transactions – that is, total withdrawals + total deposits),
        \item d. Subtracting this federal tax from the new balance.

    \end{enumerate}

After these calculations, print the final end-of-month balance.

\end{enumerate}
\end{document}
