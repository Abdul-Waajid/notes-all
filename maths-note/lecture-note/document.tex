\documentclass[]{book}
\title{Math for Computing}
\author{Abdul Waajid}
\date{10/7/2024}

%hypperref setup.
\usepackage{hyperref}
\hypersetup{
    colorlinks=true,
    linkcolor=blue,
    filecolor=magenta,      
    urlcolor=cyan,
    pdfpagemode=FullScreen,
    }
%hypperref setup.

\urlstyle{same}
\usepackage{parskip}
\usepackage{amsmath, amssymb, amsfonts}
\usepackage{graphicx}
\usepackage[top=1in,bottom=1.0in,left=0.5in,right=0.5in]{geometry}
\usepackage{enumerate}

\usepackage{float}

\setlength{\parskip}{0.3 em}
\setlength{\parskip}{1em}  % Add 1em space between paragraphs
\setlength{\parindent}{0pt} % Disable paragraph indentation

\begin{document}
\maketitle


\section{Sequences }


A sequence is an ordered list of elements (numbers, objects, ect.), usually arranged according to a certain rule or pattern.

Each element in a sequence is called a term, and the position of a term in a sequence is called an index.

\vspace{0.5 cm}
\textit{Examples}

$(2, 4, 6, 8, 10, 12)$ - sequence of even numbers

$(1, 3, 5, 7, 9, 11)$ - sequence of odd numbers

Sequences can be finite or \href{https://chatgpt.com/share/67036ea9-4ee4-8013-807b-503ab41d091b}{(Relatively)} infinite.

\subsection{Arithmetic sequence}

An arithmetic sequence (progression) is a sequence of numbers in which the difference between consecutive terms is constant,
basically an arithmetic sequence is a list of numbers that increase or deacrease by the same value every single time. 
This (constant) difference is called the "comman 
difference", denoted by d. 

\textbf{The general form of an arithmetic sequence is}

$$ a_{n} = a + (n-1)d$$

Where $ a_{n}$ is the term that you are finding.

$a$ is the first term of the sequence.

$$ a, (a+d), (a+2d), (a+3d), ...  $$  

$n$ is the position of the term in the sequence (for example the first term in the below sequence is $2$ and the term is $11$.)

\vspace{0.5 cm}
\textit{Example}: 

In the sequence $2, 5, 8, 11 ...$.  

The first term is $ 2  $, $ a = 2  $

The comman difference is $ 3  $,meaning  $ d = 3  $

Therefore to find the 4th term of this sequence using the equation $ a_{n} = a + (n-1)d$

\begin{align}  
    a_{n} &= a + (n-1) \cdot d  \\
    a_{4} &= 2 + (4-1) \cdot  3  \\
    a_{4} &= 2 + (3) \cdot 3   \\
    a_{4} &= 2 + 6  \\
    a_{4} &= 8  \\
\end{align}  


\vspace{0.5 cm}
\textit{Example}: 

In a sequence $ 10, 7, 4, 1, ...  $ find the 11th term.
 
The first term (a) is 10

The comman difference (d) is 3.

The term $ n  $ that must be found is 11.

To find the 11th term.

\begin{align}
    a_{n} &= a + (n-1) \cdot d \\
    a_{11} &= 10 + (11-1) \cdot (-3 )\\
    a_{11} &= 10 + (10) \cdot (-3) \\
    a_{11} &= 10 + (-30) \\
    a_{11} &= -20 \\
\end{align}

Remmber if the comman difference is positive the numbers get larger and if the comman difference is negative then the numbers get smaller.

It is possible to find the value of any term in a arithmetic sequence (where the numbers increase or deacrease by the same value)
with this formula.

\subsection{Geometric sequences}

A geometric sequence is a list of numbers where each term is found by multiplying the previous term by exactly the same amount
 (which cannot be zero) which is called the comman ratio " $ r  $" and the exception is the first term of the sequence.   

 How it works is that you start with the first term which will be called "$ a  $". 
Then to get the next term you multiply the first term by the comman ratio which will be called  "$ r  $", now you have the second term.
After this all you need to do is to keep multiplying the result with the comman ratio $ r  $ to get the following terms.

\textbf{Formula for the  $ n th  $ term of a geometric sequence}

$$ a_{n} = a \cdot r^{n-1}$$

 $ a_{n} $ is the term you are looking for.

 $ a  $ is the first term.

 $ r $ is the comman ratio (the number that you multiply by). 

 $ n $ is the position of the term (1st, 2nd, 3rd, etc.).

\vspace{0.5 cm}
\textit{Example}

Consider the sequence:  $ 3, 6, 12, 24, ...  $

The first term  $a$ is 3.
 
The comman ratio  $ r  $ is 2 because each term is equal to the product of the previous term by two (except the first term). 

To find the fourth term:

\begin{align}
    a_{n} &= a \cdot r^{(n-1)} \\
    a_{4} &= 3 \cdot 2^{(4-1)} \\
    a_{4} &= 3 \cdot 2^{3} \\
    a_{4} &= 3 \cdot 8 \\
    a_{4} &= 24 \\
\end{align}

So therefore the fourth term is 24 which matches the sequence.

\vspace{0.5 cm}
\textit{Example}

Another sequence: (16, 8, 4, 2, ...)

To find the fifth term of this sequence.

\begin{align}
    a_{n} &= a \cdot r^{(n-1)} \\
    a_{5} &= 16 \cdot 0.5^{(5-1)} \\
    a_{5} &= 16 \cdot 0.5^{(4)} \\
    a_{5} &= 16 \cdot 0.0625 \\
    a_{5} &=1 \\
\end{align}
 
\section{Sets}

A set is a group or collection of things that are clearly defined and distinct. These things can be anything like numbers objects or even ideas.
Think of it as 
a box that holds certain items, and only those specific items belong to that box.

Clearly defined means that we can wether an object belongs to a set or not. For example a set of all even numbers is well defined because it is 
clearly know which nmubers are part of this set.

Distinct means that each object is unique, no repeated objects within an set.

The objects that are part of a set are called the elements or members of that set. 


Sets are also usually named with capital letters, and the items of the set are written within curly brackets.

\vspace{0.5 cm}
\textit{Example}


If you have a set called $A$ that contains the numbers $1$, $2$, and $3$ you would write it like this:

\begin{center}
$A=\{1,2,3\}$
\end{center}

This means that the set $A$ contains the numbers $1$, $2$, and $3$, and only those numbers.


\vspace{0.5 cm}
\textit{Examples of sets}
\vspace{0.5 cm}

\begin{center}
Set of numbers: $\{1, 2, 3, 4\}$.
\end{center}

It is said that the numbers 1, 2. 3 and 4 are elements of the above set. 


\vspace{0.5 cm}

\begin{center}
Set of fruits: \{ "dates", "raisins", "olives"\}.

\end{center}

In this set "dates", "raisins" and "olives" are the elements.


\vspace{0.5 cm}

\begin{center}
Set of letters: \{A, B, C, D\}.
\end{center}

The letters A, B, C, D are the elements of the set.


To reiterate a point mentioned before, a set is usually written with curly braces \{\}. 
For example, if a set containing the numbers 2, 4 and 6 is to be written, it would be written as: 

$$A = \{2, 4, 6\}$$


\vspace{0.5 cm}
If an element is part of a set, we use the symbol "$\in$" (called "belongs to").

\vspace{0.5 cm}
\textit{Example}

If the number 4 is part of the set $A = \{2, 4, 6\}$, It is to be written:

$$ 4 \in A$$

Meaning that 4 is an element of set $ A$.
 
\vspace{0.5 cm}
If an element is not part of the set, the symbol "$\notin$" (does not "belong to").

\vspace{0.5 cm}
\textit{Example}

In the above set the element 5 is not part of it, therefore it is said:


$$5 \notin A$$






\end{document}
