
\documentclass[]{book}
\author{Abdul Waajid}
\date{11th of October 2024}

%hypperref setup.
\usepackage{hyperref}
\hypersetup{
    colorlinks=true,
    linkcolor=blue,
    filecolor=magenta,      
    urlcolor=cyan,
    pdfpagemode=FullScreen,
    }
%hypperref setup.

\urlstyle{same}
\usepackage{parskip}
\usepackage{amsmath, amssymb, amsfonts}
\usepackage{graphicx}
\usepackage[top=1in,bottom=1in,left=0.5in,right=0.5in]{geometry}
\usepackage{enumerate}
\usepackage{array}
\usepackage{tipa}
\usepackage{multirow} %for merging in tables
\usepackage{tabularx}
\usepackage{float}

\setlength{\parskip}{0.3 em}
\setlength{\parskip}{1em}  % Add 1em space between paragraphs
\setlength{\parindent}{0pt} % Disable paragraph indentation
\begin{document}
\section{Real Values in Binary}

\subsection{Bicimal}

Binary format for representing fractional values. Because not all values are whole numbers there is a way to represent fractions (values with a decimal)
place.


\textit{- Fixed point}

\begin{table}[H]
    \begin{center}
        \begin{tabular}{|c|c|c|c|}
            \hline
            \multicolumn{4}{|c|}{Bicimal}\\ \hline
            $2^{-1}$ & $2^{-2}$ & $2^{-3}$ & $2^{-4}$\\ \hline
            $\frac{1}{2}$ & $\frac{1}{4}$ & $\frac{1}{8}$ &  $\frac{1}{16}$\\ \hline
            $0.5$ & $0.25$ & $0.125$ & $0.0625$\\ \hline
        \end{tabular}
    \end{center}
    \caption{}
\end{table}



\end{document}


