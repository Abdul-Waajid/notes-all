\documentclass[]{article}
\title{Math for Computing}
\author{Abdul Waajid}
\date{10/7/2024}

%hypperref setup.
\usepackage{hyperref}
\hypersetup{
    colorlinks=true,
    linkcolor=blue,
    filecolor=magenta,      
    urlcolor=cyan,
    pdfpagemode=FullScreen,
    }
%hypperref setup.

\urlstyle{same}
\usepackage{parskip}
\usepackage{amsmath, amssymb, amsfonts}
\usepackage{graphicx}
\usepackage[top=1in,bottom=1in,left=0.5in,right=0.5in]{geometry}
\usepackage{enumerate}

\usepackage{float}

\setlength{\parskip}{0.3 em}
\setlength{\parskip}{1em}  % Add 1em space between paragraphs
\setlength{\parindent}{0pt} % Disable paragraph indentation

\begin{document}
\maketitle


\section{Sequences and series}

\subsection{Sequences}

A sequence is an ordered list of elements (numbers, objects, ect.), usually arranged according to a certain rule or pattern.

Each element in a sequence is called a term, and the position of a term in a sequence is called an index.

\textbf{\textit{Examples}}

$(2, 4, 6, 8, 10, 12)$ - sequence of even numbers

$(1, 3, 5, 7, 9, 11)$ - sequence of odd numbers

Sequences can be finite or \href{https://chatgpt.com/share/67036ea9-4ee4-8013-807b-503ab41d091b}{(Relatively)} infinite.

\subsubsection{Arithmetic sequence}

An arithmetic sequence  is a sequence of numbers in which the difference between consecutive terms is constant. This difference is called the "comman 
difference", denoted by d. The general form of an arithmetic sequence is 

$$ a, a+d, a+2d, a+3d, ...  $$ here, a is the first term of the sequence. 

The $nth$ term of an arithmetic sequence can be calculated using the formula: 

$$ a_{n} = a + (n-1)d$$


\end{document}
