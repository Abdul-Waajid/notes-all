\documentclass[]{article}
\title{Math for Computing}
\author{Abdul Waajid}
\date{10/7/2024}

%hypperref setup.
\usepackage{hyperref}
\hypersetup{
    colorlinks=true,
    linkcolor=blue,
    filecolor=magenta,      
    urlcolor=cyan,
    pdfpagemode=FullScreen,
    }
%hypperref setup.

\urlstyle{same}
\usepackage{parskip}
\usepackage{amsmath, amssymb, amsfonts}
\usepackage{graphicx}
\usepackage[top=1in,bottom=1in,left=0.5in,right=0.5in]{geometry}
\usepackage{enumerate}
\usepackage{tipa}
\usepackage{float}

\setlength{\parskip}{0.3 em}
\setlength{\parskip}{1em}  % Add 1em space between paragraphs
\setlength{\parindent}{0pt} % Disable paragraph indentation

\begin{document}
\maketitle

\section{Tutorial One}

\begin{enumerate}

    \item Define, using set builder notation, the set $C$ which is obtained via the following operations to sets $A$ and $B$:
    \begin{enumerate}[i.]
        \item $C = A \cap B$ 
        \item $C = A \cup B$
        \item $A$\textbackslash $B$ 
        \item $A'$  
    \end{enumerate}

    \item Let $A$ and $B$ be the only sets in $\mathbb{U}$ and $A = {5, 6, 10, 12}$ and $B = {5, 7, 11}$ Apply the following operations to sets $A$ and $B$


    \item Below $\mathbb{U}$ is the universal set, ${}$ is the empty set and A is an arbitrary set. Based on the definition of the empty and universal sets establish what should be the resulting set of the following operations:
        

    \item Consider sets $A = {a}$, $B = {g, h, i, j}$ and $C = {i, j, k, l}$. In the Venn diagram below place the elements of the following sets and establish what are the sets resulting in the following operations:

    \begin{enumerate}[i.]
        \item $B \cap C = \{j, i\} $
        \item $A \cup B = \{a, g, h, i, j\}$
        \item $A \cap B = \{ \}$
        \item $B \cup (B \cap C) = \{g, h, i, j\}$
        \item $A \cap (B$\textbackslash$ \mathbb{U}) = \{ \}$
        \item $\mathbb{U}(A\cup B) = \{a, g, h, i, j\}$
    \end{enumerate}
    \item $Let A = \{a, b, c\}$ and $B = \{1, 0\}.$
        
    \begin{enumerate}[i.]
        \item Write down all elements of the Power Set of A and Power Set of B.
            
            $P(A)= \{\{ \}, \{a\}, \{b\}, \{c\}, \{a, b\}, \{a, c\}, \{a, b, c\} \}$

            $P(B)= \{ \{ \}, \{1\}, \{0\}, \{1, 0\} \}$

        \item List all the elements of $A \times B$

            $A \times B = \{ (a, 1), (a, 0), (b, 1), (b, 0), (c, 1), (c, 0) \}$

        \item List all the elements of $B \times A$

            $B \times A = \{ (1, a), (0, a), (1, b), (0, b), (1, c), (0, c) \}$

    \end{enumerate}
\item CHALLENGE

    \begin{enumerate}[i.]
        \item Show that if $A \subseteq B$ andi $ B \subseteq A$ then $ A = B$, i.e. that $A$ and $B$ are equivalent, i.e. they have the same elements.
        \item Let $A = [2,9]$ be a closed interval of all real numbers 2 to 9 Which interval is introduced by the following set builder notation:$ \{x\in A : \sqrt[2]{x} \in A$ \& $ x < \sqrt[2]{20} \}$.
    \end{enumerate}

\end{enumerate}



\end{document}









