\documentclass[]{article}
\title{Math for Computing}
\author{Abdul Waajid}
\date{10/7/2024}

%hypperref setup.
\usepackage{hyperref}
\hypersetup{
    colorlinks=true,
    linkcolor=blue,
    filecolor=magenta,      
    urlcolor=cyan,
    pdfpagemode=FullScreen,
    }
%hypperref setup.

\urlstyle{same}
\usepackage{parskip}
\usepackage{amsmath, amssymb, amsfonts}
\usepackage{graphicx}
\usepackage[top=1in,bottom=1in,left=0.5in,right=0.5in]{geometry}
\usepackage{enumerate}

\usepackage{float}

\setlength{\parskip}{0.3 em}
\setlength{\parskip}{1em}  % Add 1em space between paragraphs
\setlength{\parindent}{0pt} % Disable paragraph indentation

\begin{document}
\maketitle


\section{Sequences and series}

\subsection{Sequences}

A sequence is an ordered list of elements (numbers, objects, ect.), usually arranged according to a certain rule or pattern.

Each element in a sequence is called a term, and the position of a term in a sequence is called an index.

\textbf{\textit{Examples}}

$(2, 4, 6, 8, 10, 12)$ - sequence of even numbers

$(1, 3, 5, 7, 9, 11)$ - sequence of odd numbers

Sequences can be finite or \href{https://chatgpt.com/share/67036ea9-4ee4-8013-807b-503ab41d091b}{(Relatively)} infinite.

\subsubsection{Arithmetic sequence}

An arithmetic sequence (progression) is a sequence of numbers in which the difference between consecutive terms is constant,
basically an arithmetic sequence is a list of numbers that increase or deacrease by the same value every single time. 
This (constant) difference is called the "comman 
difference", denoted by d. The general form of an arithmetic sequence is 


$$ a_{n} = a + (n-1)d$$

Where $ a_{n}$ is the term that you are finding.

$a$ is the first term of the sequence.

$$ a, (a+d), (a+2d), (a+3d), ...  $$  

$n$ is the position of the term in the sequence (for example the first term in the below sequence is $2$ and the term is $11$.)

\vspace{0.5 cm}
\textit{Example}: 

In the sequence $2, 5, 8, 11 ...$.  

The first term is $ 2  $, $ a = 2  $

The comman difference is $ 3  $,meaning  $ d = 3  $

Therefore to find the 4th term of this sequence using the equation $ a_{n} = a + (n-1)d$

\begin{align}  
    a_{n} &= a + (n-1) \cdot d  \\
    a_{4} &= 2 + (4-1) \cdot  3  \\
    a_{4} &= 2 + (3) \cdot 3   \\
    a_{4} &= 2 + 6  \\
    a_{4} &= 8  \\
\end{align}  


\textit{Example}: 

In a sequence $ 10, 7, 4, 1, ...  $ find the 11th term
 
The first term (a) is 10 

The comman difference (d) is 3

The term $ n  $ that must be found is 11

To find the 11th term

\begin{align}
    a_{n} &= a + (n-1) \cdot d \\
    a_{11} &= 10 + (11-1) \cdot (-3 )\\
    a_{11} &= 10 + (10) \cdot (-3) \\
    a_{11} &= 10 + (-30) \\
    a_{11} &= -20 \\
\end{align}

Remmber if the comman difference is positive the numbers get larger and if the comman difference is negative then the numbers get smaller.

It is possible to find the value of any term in a arithmetic sequence (where the numbers increase or deacrease by the same value)
with this formula.

\subsubsection{Geometric sequences}

A geometric sequence (progression)




\end{document}

